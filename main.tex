\documentclass{amsart}
\usepackage[ocgcolorlinks,linktoc=all]{hyperref}
\usepackage{cite}
%\usepackage{cancel}
\hypersetup{citecolor=blue,linkcolor=red}
\newtheorem{theorem}{Theorem}
\newtheorem*{thmA}{Theorem}
\newtheorem*{thmB}{Lemma}
\newtheorem*{claim}{Claim}
\newtheorem*{rem}{Remark}
\newtheorem*{thmmain}{Theorem}
\newtheorem{lemma}[theorem]{Lemma}
\newtheorem{proposition}[theorem]{Proposition}
\newtheorem*{propmain}{Proposition}
\newtheorem{corollary}[theorem]{Corollary}
\theoremstyle{definition}
\newtheorem{definition}[theorem]{Definition}
\newtheorem{example}[theorem]{Example}
\newtheorem{xca}[theorem]{Exercise}

\theoremstyle{remark}
\newtheorem{remark}[theorem]{Remark}
\def\Boom{{\Box}}
\newcommand{\abs}[1]{\lvert#1\rvert}
\numberwithin{equation}{section}

\newcommand{\blankbox}[2]{%
  \parbox{\columnwidth}{\centering
    \setlength{\fboxsep}{0pt}%
    \fbox{\raisebox{0pt}[#2]{\hspace{#1}}}%
  }%
}

\begin{document}

\title[]
 {A unified flow approach to smooth, even $L_p$-Minkowski problems}

\author[P. Bryan]{Paul Bryan}
\address{Mathematics Institute, University of Warwick
Coventry, CV4 7AL, England}
\email{p.bryan@warwick.ac.uk}
\author[M.N. Ivaki]{Mohammad N. Ivaki}
\address{Institut f\"{u}r Diskrete Mathematik und Geometrie, Technische Universit\"{a}t Wien,
Wiedner Hauptstr. 8--10, 1040 Wien, Austria}
\email{mohammad.ivaki@tuwien.ac.at}
\author[J. Scheuer]{Julian Scheuer}
\address{Albert-Ludwigs-Universit\"{a}t,
Mathematisches Institut, Eckerstr. 1, 79104
Freiburg, Germany}
\email{julian.scheuer@math.uni-freiburg.de}
\dedicatory{}
\subjclass[2010]{}
\keywords{}

\begin{abstract}
We study the asymptotic behavior of a family of curvature flows. Among other results, we give a unified flow approach to the existence of smooth, even $L_p$-Minkowski problems in $\mathbb{R}^{n+1}$ for $p>-n-1.$
\end{abstract}

\maketitle
\section{Introduction}
The Minkowski problem deals with existence, uniqueness, regularity, and stability of closed convex hypersurfaces whose Gauss curvature (as a function of the outer normals) is preassigned. Major contributions to this problem were made by Minkowski \cite{M1,M2}, Aleksandrov \cite{A2,A3,A4}, Fenchel and Jessen \cite{FJ}, Lewy \cite{Le1,Le2}, Nirenberg \cite{N}, Calabi \cite{Cal}, Pogorelov \cite{P1,P2}, Cheng and Yau \cite{ChYau}, Caffarelli, Nirenberg, and Spruck \cite{CNS}, and others. A generalization of the Minkowski problem known as the $L_p$-Minkowski problem was introduced by Lutwak \cite{Lu1,Lu2}. This generalization was studied in \cite{Lu1,LuO}. Solutions to many cases of these generalized problems followed later in \cite{1,AnCrys,Andrews Ben 2000,Andrews 2003,27,36,39,49,50,51,72,LYZ,79,104,110,s1,s2,Zhu1,Zhu2,Zhu3,jiang,QL,DZ}.

In the smooth category, the $L_p$-Minkowski problem asks given a smooth, strictly positive function $g : \mathbb{S}^n \to \mathbb{R}$, does there exists a smooth, closed, strictly convex hypersurface $M_0$ such that,
\begin{equation}
\label{e:lp}
\frac{1}{\mathcal{K}(x)} = h^{p-1}(\nu(x)) g(\nu(x))
\end{equation}
where $x \in M_0$, $h$ denotes the support function, $\mathcal{K}$ the Gauss curvature and $\nu$ the Gauss map $M_0 \to \mathbb{S}^n$. The \emph{even} $L_p$-Minkowski problem requires in addition, that $g$ is an even function. The case $p=1$, is the original Minkowski problem.


The existence and regularity of solutions to the $L_p$-Minkowski problem are discussed in \cite{39} for $p > −n-1$; however, the methods used in \cite{39} vary significantly depending on $p$. Our study on (\ref{e: flow0}) below is motivated by the search for a variational proof (based on curvature flow) of the $L_p$-Minkowski problem. For $p=1$, Chou-Wang \cite{Chou Wang 2000} treated the classical $L_1$-Minkowki problem in the smooth category by a logarithmic Gauss curvature flow.   For $n=1$, and $p\neq 1>-3$, the existence of solutions to the $L_p$-Minkowski problems follows from Andrews' results \cite{Andrews 1998} on the asymptotic behavior of a family of contracting and expanding flows of curves. Also, in higher dimensions, the existence of solutions to the $L_p$-Minkowski problems follows from \cite{Andrews Ben 2000} when $-n-1<p\leq -n+1$ (a short proof of this is also given in \cite{Ivaki-Proc}) or when $\varphi$ is even (e.q., $\varphi(u)=\varphi(-u)$) and $-n+1<p<1.$ See also \cite{Andrews 1999,AGN, GN, U1, U2}.


For convenience, in studying the flow we consider $\varphi = 1/g$. Consider a smooth, closed, strictly convex hypersurface $M_0$ in Euclidean space $\mathbb{R}^{n+1}$ given by a smooth embedding $F_0:M\to \mathbb{R}^{n+1}.$ Suppose the origin is in the interior of the region enclosed by $M_0.$ We study the long-time behavior of a family of hypersurfaces $\{M_t\}$ given by smooth maps $F:M\times [0,T)\to \mathbb{R}^{n+1}$ satisfying the initial value problem
\begin{equation}\label{e: flow0}
 \partial_{t}F(x,t)=\varphi(\nu(x,t))\frac{(F(x,t)\cdot \nu (x,t))^{2-p}}{\mathcal{K}(x,t)} \nu(x,t),\quad
 F(\cdot,0)=F_{0}(\cdot).
\end{equation}
Here $\mathcal{K}(\cdot,t)$ and $\nu(\cdot,t)$ are the Gauss curvature and the outer unit normal vector of $M_t=F(M,t)$ and $\varphi$ is a positive, smooth function on $\mathbb{S}^{n}$. Furthermore, $T$ is the maximal time for which the solution exists.


For $p=2,~\varphi\equiv1$, flow (\ref{e: flow0}) was studied by Schn\"{u}rer \cite{Oliver 2006} in $\mathbb{R}^3,$ and by Gerhardt \cite{Gerhardt 2014} in higher dimensions. Both works rely on the reflection principle of Chow and Gulliver \cite{Bennett Chow and Robert Gulliver 1996}, and McCoy \cite{James A. McCoy 2003}. Their result is as follows: the volume-normalized flow evolves any $M_0$ in the $C^{\infty}$-topology to an origin-centered ball. For $p>2, \varphi\equiv1$ and in $\mathbb{R}^{n+1}$, it follows from Chow-Gulliver \cite[Theorem 3.1]{Bennett Chow and Robert Gulliver 1996} (see also Tsai \cite[Example 1]{Tsai 2005}) that (\ref{e: flow0}) evolves $M_0$, after rescaling to fixed volume, in the $C^{1}$-topology to an origin-centered ball. We refer the reader to the paper \cite{Ivaki 2014-gauss} regarding a rather comprehensive list of previous works on this curvature flow.

Let us set
$\tilde{K_t}=(V(B)/V(K_t))^{1/(n+1)}K_t,$ where $K_t$ denotes the convex body enclosed by $M_t$ and $V(K_t)$ is the $(n+1)$-dimensional Lebesugue measure of $K_t.$ The following theorem was proved in \cite{Ivaki 2014-gauss} regarding the case $p=-n-1,~\varphi\equiv1$ (in this case the flow belonges to a family of centro-affine normal flows introduced by Stancu in \cite{Alina 2012}):
\begin{thmmain}[\cite{Ivaki 2014-gauss}]
Let $n\ge2,$ $p=-n-1,~\varphi\equiv1$ and suppose $K_0$ has its Santal\'{o} point at the origin, e.q., $\int_{\mathbb{S}^{n}}\frac{u}{h_{K_0}(u)^{n+2}}d\sigma(u)=0$. Then there exists a unique solution $\{K_t\}$ of flow (\ref{e: flow0}), such that $\tilde{K}_t$ converges in $C^{\infty}$ to an origin-centered ellipsoid.
\end{thmmain}
Here $h_{K_0}$ is the support function of $K_0.$ A closed, convex hypersurface $M_0$ can be described in terms of its support function $h_{K_0}:\mathbb{S}^n\to\mathbb{R}$ defined by
\[h_{K_0}(u)=\sup\{u\cdot x: x\in M_0\}.\]
If $M_0$ is smooth and strictly convex, then $h_{K_0}(u)=u\cdot F_0(\nu^{-1}(u)).$


From the evolution equation of $F(\cdot,t)$ it follows that $$h(\cdot,t):=h_{K_t}(\cdot):\mathbb{S}^{n}\times [0,T)\to \mathbb{R}$$ evolves by
\begin{equation}\label{eq: flow4}
\partial_th(u,t)=\varphi(u)(h^{2-p}S_{n})(u,t),
\end{equation}
where $S_{n}(u,t)=1/\mathcal{K}(\nu^{-1}(u,t),t).$
A homothetic self-similar solution of this flow satisfies
\begin{align} \label{def: self similar}
h^{1-p} \det (\bar{\nabla}^2 h + \operatorname{Id}h)=\frac{c}{\varphi},
\end{align}
for some positive constant $c.$ Here $\bar{\nabla}$ is the covariant derivative on $\mathbb{S}^{n}$ endowed with an orthonormal frame. Note that $S_{n}=\det (\bar{\nabla}^2 h + \operatorname{Id}h).$  Thus self-similar solutions to the flow \eqref{e: flow0} are precisely solutions of the $L_p$-Minkowksi problem, \eqref{e:lp}. To obtain such solutions we prove the following theorems:

\begin{theorem}[Even $L_p$-Minkowski problem]
Let $-n-1<p<\infty$ and $\varphi$ be a positive, smooth even function on $\mathbb{S}^{n}$ i.e., $\varphi(u)=\varphi(-u)$. Suppose $K_0$ is origin-symmetric. There exists a unique origin-symmetric solution $\{K_t\}$ of (\ref{e: flow0}) such that $\{\tilde{K}_t\}$ converges for a subsequence of times in $C^{1}$ to a smooth, origin-symmetric, strictly convex solution of (\ref{def: self similar}). Also, when $p\leq n+1$ the convergence is in $C^{\infty}$, and if $p\ge 1$ the convergence holds for the full sequence.
\end{theorem}
The next theorem establishes the existence of solutions to the smooth $L_p$-Minkowski problem for $-n-1< p\leq -n$:
\begin{theorem}
Let $-n-1< p\leq -n$ and $K_0$ satisfy $\int_{\mathbb{S}^{n}}\frac{u}{\varphi(u)h_{K_0}(u)^{1-p}}d\sigma(u)=0$. There exists a unique solution $\{K_t\}$ of flow (\ref{e: flow0}) such that $\{\tilde{K}_t\}$ converges for a subsequence of times in $C^{\infty}$ to a positive, smooth, strictly convex solution of (\ref{def: self similar}).
\end{theorem}
We also prove following theorem on the asymptotic behavior of (\ref{e: flow0}) when $\varphi\equiv 1:$
\begin{theorem}
Let $p\neq 1>-n-1,~\varphi\equiv 1$ and $K_0$ satisfy $\int_{\mathbb{S}^{n}}\frac{u}{h_{K_0}(u)^{1-p}}d\sigma(u)=0$. Then there exists a unique solution $\{K_t\}$ of (\ref{e: flow0}) such that $\{\tilde{K}_t\}$ converges for a subsequence of times in $C^{1}$ to a positive, smooth, strictly convex solution of (\ref{def: self similar}). In addition, for $p\neq 1\leq n+1$ the convergence holds in $C^{\infty}$, and when $p> 1$ the full sequence converges to the unit ball.
\end{theorem}
\begin{remark}
In the previous theorem, in some other cases than $p\ge 1$, it is known that the limiting shape is the unit ball. See, for example, \cite{Andrews 1999, andrewschen}.
\end{remark}

The main difficulty to prove the convergence of the normalized solutions is the long-time existence issue. To prove long time existence, we first obtain bounds on the Gauss curvature in Section \ref{subsec:gauss-curvature-bounds} using the well known standard technique of Tso \cite{Tso} to obtain upper bounds, with lower bounds obtained by applying the same technique to the evolution of the polar body, as in \cite{Ivaki-Proc}.  Controlling the principal curvatures requires estimates of higher derivatives of the speed which is generally quite difficult due to the non-linearity of the flow. In Section \ref{subsec:principal-curvature-bounds} we obtain these crucial estimates by adapting the $C^2$ estimates of Guan-Ren-Wang for the prescribed curvature problem see \cite[(4.2)]{Guan}, and long time existence follows readily by standard arguments. Once it is proved that solutions to the flow exist until they expand to infinity uniformly in all directions, the method of \cite[Section 8]{Ivaki 2014-gauss} applies and yields the convergence of the volume-normalized solutions in $C^{1}$ to self-similar solutions provided $p\neq1$. Further work is required to establish the convergence of the normalized solutions when $p=1$ and to prove the convergence in $C^{\infty}$ if $p\leq n+1$; this is accomplished in Section \ref{sec: nor conv}; see also Remark \ref{rem}.
\section*{Acknowledgment}
The work of the second author was supported by Austrian Science Fund (FWF) Project
M1716-N25 and the European
Research Council (ERC) Project 306445.


%\begin{theorem}\label{thm: 3}
%Let $p>1,$ and $K_0\in\mathcal{F}_0^{n+1}$ satisfy $\int_{\mathbb{S}^{n}}\frac{z}{\varphi(z)u_{K_0}(z)^{1-p}}d\sigma(z)=0$. Then there exists a unique solution $\{K_t\}\subset \mathcal{F}_0^{n+1} $ of flow (\ref{eq: flow4}) such that $\{\tilde{K}_t\}$ converges in Hausdorff topology to a convex body $\tilde{K}_{\infty}\in\mathcal{K}^{n+1}$ that may have the origin of on its boundary and is the unique solution of the $L_p$-Minkowski problem.
%\end{theorem}
%\begin{remark}
%Theorem \ref{thm: 3} is the first example of a geometric flows in higher dimensions that gives a convergence result to a homothetic self-similar solution with the origin on its boundary. In dimension two such convergence result was proved by Andrews for a different curvature flow; see  \cite{Andrews 1998}.
%\end{remark}
%\begin{remark}
%If $-n-1<p\leq -n,$ for a given $\varphi$ and $K$, a Euclidean translation of $K$, $K+v$, satisfies $\int_{\mathbb{S}^{n}}\frac{z}{\varphi(z)u_{K+v}(z)^{1-p}}d\sigma(z)=0.$
%Regarding the case $p>1,$ if $\varphi^{1/(p-1)}$ is the support function of some smooth, strictly convex body then we may choose $K_0$ to be the corresponding convex body so that $\int_{\mathbb{S}^{n}}\frac{z}{\varphi(z)u_{K_0}(z)^{1-p}}d\sigma(z)=0.$
%\end{remark}


\section{basic evolution equations}

Let $g=\{g_{ij}\}$, and $W=\{w_{ij}\}$ denote, in order, the induced metric and the second fundamental form of $M$. At every point in the hypersurface $M$ choose a local orthonormal frame $\{e_1,\cdots, e_n\}.$

We use the following standard notation
\[w_i^j=g^{mj}w_{im},\]
\[(w^2)_i^j=g^{mj}g^{rs}w_{ir}w_{sm},\]
\[|W|^2=g^{ij}g^{kl}w_{ik}w_{lj}=w_{ij}w^{ij}.\]
Here, $\{g^{ij}\}$ is the inverse matrix of $\{g_{ij}\}.$


We use semicolons to denote covariant derivatives. The following geometric formulas are well-known:
\begin{align*}
\nu_{;i} &= w_i^ke_k,\\
\nu_{;ij} &= g^{kl}w_{ij;l}e_k - w_i^lw_{lj}\nu,\\
h_{;i} &= w_i^k (F\cdot e_k),\\
h_{;ij} &= w_{ij} - hw_i^lw_{lj} + F \cdot \nabla w_{ij}.\\
%\|F\|^2_{;ij}&=2g_{ij}-2w_{ij}h.
\end{align*}
Note that in above we considered the support function as a function on the boundary of the hypersurface; that is, at the point $x\in M$ we have
$$h(x)=F(x)\cdot \nu(x).$$


For convenience, let $\psi(x)=h^{2-p}(x)\varphi(\nu(x))$. The following evolution equations are standard; see, for example, \cite{Gerhardt:/2006}.
\begin{lemma} The following evolution equations hold:
\[\partial_t \nu = -\nabla \left(\frac{\psi}{\mathcal{K}}\right),\]
\begin{align*}
\partial_t w_i^j &= -\left(\frac{\psi}{\mathcal{K}}\right)_{;ij} - \left(\frac{\psi}{\mathcal{K}}\right) w_i^kw_k^j \\
&= \psi \frac{\mathcal{K}^{kl}}{\mathcal{K}^2} w_{i;kl}^j + \psi \frac{\mathcal{K}^{kl}}{\mathcal{K}^2} w_{kr}w_l^rw_i^j -(n+1)\frac{\psi }{\mathcal{K}} w_{i}^k w_k^{j} \\
&\quad  + \psi\frac{\mathcal{K}^{kl,rs}}{\mathcal{K}^{2}}g^{jm} w_{kl;i}w_{rs;m}- \frac{2\psi}{\mathcal{K}^{3}} g^{jm} \mathcal{K}_{;i} \mathcal{K}_{;m} \\
&\quad + \frac{1}{\mathcal{K}^{2}}g^{jk}\mathcal{K}_{;k}\psi_{;i} + \frac{1}{\mathcal{K}^{2}} g^{jk}\psi_{;k}\mathcal{K}_{;i} - \frac{1}{\mathcal{K}} g^{jk}\psi_{;ik},
\end{align*}
\begin{align*}
\partial_th &= \psi \frac{\mathcal{K}^{ij}}{\mathcal{K}^{2}} h_{;ij} + \psi h \frac{\mathcal{K}^{ij}}{\mathcal{K}^{2}} w_i^lw_{lj} - (n-1) \frac{\psi}{\mathcal{K}}- \frac{1}{\mathcal{K}} F\cdot\nabla\psi.
\end{align*}
%\begin{align*}
%\partial_t\|F\|^2=\psi\dot{\Phi}S^{ij}\|F\|^2_{;ij}-2h\psi\dot{\Phi}S^{ij}g_{ij}+2\psi(\dot{\Phi}S-\Phi)h.
%\end{align*}
\end{lemma}

\section{long-time existence}
\label{sec:long-time-existence}

\subsection{Lower and upper bounds on Gauss curvature}
\label{subsec:gauss-curvature-bounds}

The proofs of the following two lemmas are similar to the proofs of \cite[ Lemmas 4.1, 4.2]{Ivaki-Proc}. For completeness, we give the proofs here. In this section only we use $\bar{\nabla}$ to denote covariant derivatives on the sphere with respect to the standard metric.


The matrix of the radii of the curvature of a smooth, closed, strictly convex hypersurface is denoted by $\mathfrak{r}=[\mathfrak{r}_{ij}]$ and the entries of $\mathfrak{r}$ are considered as functions on the unit sphere. They can be expressed in terms of the support function as $\mathfrak{r}_{ij}:=\bar{\nabla}_i\bar{\nabla}_j h+\bar{g}_{ij}h,$ where $[\bar{g}_{ij}]$ is the standard metric on $\mathbb{S}^{n}$. Additionally, we recall that
$S_n=\det [\mathfrak{r}_{ij}]/\det[\bar{g}_{ij}].$
\begin{lemma}\label{lem: lower}
Let $\{K_t\}$ be a solution of (\ref{e: flow0}) on $[0,t_1]$. If $c_2\leq h_{K_t}\leq c_1$ on $[0,t_1]$, then $\mathcal{K}\leq c_4$ on $[0,t_1].$ Here $c_4$ depends on $K_0$, $c_1,c_2,p,\varphi$ and $t_1.$
\end{lemma}
\begin{proof}
We apply the maximum principle to the following auxiliary function defined on the unit sphere
\[\Theta=\frac{\psi S_n}{2c_1-h}=\frac{\partial_t h}{2c_1-h}.\]
At any minimum of $\Theta$ we have
\[0=\bar{\nabla}_i\Theta_{;i}=\bar{\nabla}_i \left(\frac{\psi S_{n}}{2c_1-h}\right)\ \ \ {\hbox{and}}\ \ \  \bar{\nabla}_i\bar{\nabla}_j \Theta\geq 0.\]
Therefore,
\[ \frac{\bar{\nabla}_i (\psi S_{n})}{2c_1-h}=-\frac{\psi S_{n} \bar{\nabla}_i h}{(2c_1-h)^2}, \] and
\begin{equation}\label{e: tso dual}
\bar{\nabla}_i\bar{\nabla}_j (\psi S_{n})+\bar{g}_{ij} \psi S_{n}\geq
\frac{-\psi S_{n}\mathfrak{r}_{ij}+2c_1\psi S_{n}\bar{g}_{ij}}{2c_1-h}.
\end{equation}
Differentiating $\Theta$ with respect to time yields
\begin{align*}
\partial_t\Theta&=\frac{\psi S_n^{ij}}{2c_1-h}
\left(\bar{\nabla}_i\bar{\nabla}_j(\psi S_{n})+\bar{g}_{ij}\psi S_{n}\right)
+\frac{\psi^2S_n^{2}}{(2c_1-h)^2}\left(1+(2-p)h^{1-p}(2c_1-h)\right),
\end{align*}
 where $S_n^{ij}$ is the derivative of $S_n$ with respect to the entry $ \mathfrak{r}_{ij}$.
By applying inequality (\ref{e: tso dual}) to the preceding identity we deduce
\begin{equation}\label{e: last step tso dual}
\partial_t\Theta\geq \Theta^2\left(1-n+2 c_1\mathcal{H}\right)-c\Theta^2.
\end{equation}
%Next, we estimate the mean curvature $\mathcal{H}=\bar{g}_{ij}\mathfrak{r}^{ij}$ from below by a negative power of $\Theta$:
%\begin{align*}
%\mathcal{H}&\geq n\left(\frac{2c_1-h}{\psi S_{n}}\right)^{\frac{1}{n}}\left(\frac{\psi}
%{2c_1-h}\right)^{\frac{1}{n}}\\
%&\geq n\Theta^{-\frac{1}{n}} \left(\frac{\min \psi}{2c_1-c_2}\right)^{\frac{1}{n}}.
%\end{align*}
%Consequently, inequality (\ref{e: last step tso dual}) can be rewritten as follows
%\begin{align*}
%\partial_t\Theta&\geq \Theta^2\left(-c+2nc_1\Theta^{-\frac{1}{n}} \left(\frac{\min \psi}{2c_1-c_2}\right)^{\frac{1}{n}}\right)\\
%&=\Theta^2\left(-c+c'(c_1,c_2,\varphi)\Theta^{-\frac{1}{n}}\right)\geq -c\Theta^2.
%\end{align*}
Therefore,
\[\partial_t\frac{\varphi\frac{h^{2-p}}{\mathcal{K}}}{2c_1-h}\geq -c\left(\frac{\varphi\frac{h^{2-p}}{\mathcal{K}}}{2c_1-h}\right)^2,\]
and \[\frac{\varphi\frac{h^{2-p}}{\mathcal{K}}}{2c_1-h}(t, u)\geq \frac{1}{ct+1/\min\limits_{u\in\mathbb{S}^{n}}\frac{\varphi\frac{h^{2-p}}{\mathcal{K}}}{2c_1-h}(0, u)}\geq \frac{1}{ct_1+1/\min\limits_{u\in\mathbb{S}^{n}}\frac{\varphi\frac{h^{2-p}}{\mathcal{K}}}{2c_1-h}(0, u)}.\]
\end{proof}
\begin{lemma}\label{lem: upper}
Let $\{K_t\}$ be a solution of (\ref{e: flow0}) on $[0,t_1]$. If $c_1\leq h_{K_t}\leq c_2$ on $[0,t_1]$, then $\mathcal{K}\geq \frac{1}{a+b t^{-\frac{n}{n+1}}}$ on $(0,t_1],$ where $a$ and $b$ depend only on $c_1,c_2,p,\varphi.$ In particular, $\mathcal{K}\ge c_3$ on $[0,t_1]$ for some positive  number that depends on $K_0$, $c_1,c_2,p,\varphi$ and is independent of $t_1.$
\end{lemma}
\begin{proof}
Suppose $K_t^{\ast}$ is the polar body\footnote{The polar body of convex body $K$ with the origin of $\mathbb{R}^{n+1}$ in its interior is the convex body defined as
\[K^{\ast}=\{x\in\mathbb{R}^{n+1}| x\cdot y\leq 1 \mbox{~for~all~}y\in K\}.\]} of $K_t$ with respect to the origin. We furnish quantities associated with polar bodies with $^\ast$. The polar bodies evolve by
$$\partial_th^{\ast}=-\psi^{\ast} S_{n}^{\ast-1},\quad h^{\ast}(\cdot,t)=h_{K_{t}^{\ast}}(\cdot),$$
where $$\psi^{\ast}=\frac{(h^{\ast2}+|\bar{\nabla}h^{\ast}|^2)^{\frac{n+1+p}{2}}}{h^{\ast n+1}}\varphi\left(\frac{h^{\ast}u+\bar{\nabla} h^{\ast}}{\sqrt{h^{\ast2}+|\bar{\nabla}h^{\ast}|^2}}\right);$$
see Lemma \ref{app1} for the proof.
In addition, we have $c_1'=1/c_2\leq h^{\ast}\leq 1/c_1=c_2'.$ We will show  the function
\[\Theta=\frac{\psi^{\ast} S_{n}^{\ast-1}}{h^{\ast}-c_1'/2}\]
remains bounded.
At any point maximum of $\Theta:$
\[0=\bar{\nabla}_i\Theta=\bar{\nabla}_i \left(\frac{\psi^{\ast} S_{n}^{\ast-1}}{h^{\ast}-c_1'/2}\right)\ \ \ {\hbox{and}}\ \ \  \bar{\nabla}_i\bar{\nabla}_j \Theta\leq 0.\]
Hence, we obtain
\begin{equation}\label{e: grad of Q}
\frac{\bar{\nabla}_i (\psi^{\ast} S_{n}^{\ast-1})}{h^{\ast}-c_1'/2}=\frac{\psi^{\ast} S_{n}^{\ast-1} \bar{\nabla}_i h^{\ast}}{(h^{\ast}-c_1'/2)^2},
\end{equation}
and consequently,
\begin{equation}\label{e: tso}
\bar{\nabla}_i\bar{\nabla}_j(\psi^{\ast} S_{n}^{\ast-1})+\bar{g}_{ij}\psi^{\ast} S_{n}^{\ast-1}\leq
\frac{\psi^{\ast} S_{n}^{\ast-1}\mathfrak{r}_{ij}^{\ast}-c_1'/2\psi^{\ast} S_{n}^{\ast-1}
\bar{g}_{ij}}{h^{\ast}-c_1'/2}.
\end{equation}
Differentiating $\Theta$ with respect to time yields
\begin{align*}
\partial_t\Theta=&\frac{\psi^{\ast}S_n^{\ast -2}}{h^{\ast}-c_1'/2}S_n^{\ast ij}
\left(\bar{\nabla}_i\bar{\nabla}_j(\psi^{\ast} S_{n}^{\ast-1})+\bar{g}_{ij}
\psi^{\ast} S_{n}^{\ast-1}\right)\\
&+\frac{S_{n}^{\ast-1}}{h^{\ast}-c_1'/2}\partial_t \psi^{\ast}+\Theta^2.
\end{align*}
On the other hand, since
\[|\partial_t h^{\ast}|=\psi^{\ast} S_{n}^{\ast-1},\quad \|\bar{\nabla}\partial_t h^{\ast}\|=\|\bar{\nabla}(\psi^{\ast} S_{n}^{\ast-1})\|=\frac{\psi^{\ast} S_{n}^{\ast-1} \|\bar{\nabla} h^{\ast}\|}{h^{\ast}-c_1'/2},\quad\|\bar{\nabla} h^{\ast}\|\leq c_2',\]
where for the second equation we used (\ref{e: grad of Q}),  we have
\[\frac{S_{n}^{\ast-1}}{h^{\ast}-c_1'/2}\partial_t \psi^{\ast}\leq c(n,p,c_1,c_2,\varphi)\Theta^2.\]
Employing this last inequality and inequality (\ref{e: tso}) we infer that, at any point where the maximum of $\Theta$ is reached, we have
\begin{equation}\label{e: last step tso}
\partial_t\Theta\leq\Theta^2\left(c'-\frac{c_1'}{2}\mathcal{H}^{\ast}\right).
\end{equation}
Moreover,
\begin{align*}
\mathcal{H}^{\ast}&\geq n\left(\frac{h^{\ast}-c_1'/2}{\psi^{\ast} S_{n}^{\ast-1}}\right)^{-\frac{1}{n}}
\left(\frac{\psi^{\ast}}{h^{\ast}-c_1'/2}\right)^{-\frac{1}{n}}\\
&\geq n\Theta^{\frac{1}{n}} \left(\frac{c''}{c_1'-c_1'/2}\right)^{-\frac{1}{n}}.
\end{align*}
Therefore, we can rewrite the inequality (\ref{e: last step tso}) as follows
\begin{align*}
\partial_t\Theta&\leq \Theta^2\left(c-c'\Theta^{\frac{1}{n}} \right),
\end{align*} for positive constants $c$ and $c'$ depending only on $p,c_1,c_2,\varphi.$
Hence,
\begin{equation}\label{ie: upper on psi}
\Theta\leq c+c't^{-\frac{n}{n+1}}
\end{equation}
for some positive constants depending only on $p,c_1,c_2,\varphi.$
\footnote{
\begin{claim}Suppose $f$ is a positive smooth function of $t$ on $[0,t_1]$ that satisfies
\begin{align}\label{claim}
\frac{d}{dt}f\leq c_0+c_1f+c_2f^2-c_3f^{2+p},
\end{align}
where $c_3,p$ are positive. There exists constant $c,c'>0$ independent of the solution and depending only on $c_0,c_1,c_2,c_3,p$, such that $f\leq c+c't^{-1/(p+1)}~\mbox{on}~(0,t_1],$
\end{claim}
\begin{proof}
Note that there exists $x_0>0$ such that $c_0+c_1x+c_2x^2-c_3x^{2+p}<-c_3/2x^{2+p}$ for $x>x_0.$ If $f(0)\leq x_0$, then $f$ may increase forward in time, but when $f$ reaches $x_0$, then $f$ must start decreasing (since the right-hand side of (\ref{claim}) becomes negative).
Thus we may assume, without loss of generality, that $f(0)>x_0.$ Therefore, $f>x_0$ on a maximal time interval $[0,t_0).$ On $[0,t_0)$ we can solve
$$\frac{d}{dt}f\le-c_3/2f^{2+p}$$ to obtain
$$f\leq (c_3(p+1)/2t)^{-1/(p+1)}.$$
At $t_0$ we have $c_0+c_1f+c_2f^2-c_3f^{1+p}=-c_3/2f^{2+p}$ and $f=x_0;$ therefore the right-hand side of (\ref{claim}) is still negative. So $f\le f(t_0)$ on $[t_0,t_1].$ In conclusion, $$f\leq \max\{(c_3(p+1)/2t)^{-1/(p+1)}, x_0=f(t_0)\}\leq c+c't^{-1/(1+p)},$$ where $c,c'$ do not depend on solutions.
\end{proof}}
Now we can use the argument given in \cite[Lemma 2.3]{Ivaki-Stancu} to conclude that on $(0,t_1]$ we have
$$\mathcal{K}\geq \frac{1}{a+b t^{-\frac{n}{n+1}}}$$ for some $a$ and $b$ depending only on $c_1,c_2,p,\varphi.$
The lower bound for $\mathcal{K}$ on $[0,\delta]$ for a small enough $\delta>0$ follows from the short-time existence of the flow. The lower bound for $\mathcal{K}$ on $[\delta,t_1]$ follows from the inequality $\mathcal{K}\geq \frac{1}{a+b \delta^{-\frac{n}{n+1}}}.$
\end{proof}

\subsection{Upper and lower bounds on principal curvatures}
\label{subsec:principal-curvature-bounds}

 To obtain upper and lower bounds on the principal curvatures, we will consider the test function used by Guan-Ren-Wang for a prescribed curvature problem; see \cite[(4.2)]{Guan}.
\begin{lemma}\label{lem: final}
Let $\{K_t\}$ be a solution of (\ref{e: flow0}) on $[0,t_1]$. If $c_1\leq h_{K_t}\leq c_2$ on $[0,t_1]$, then $c_5\leq \kappa_i\le c_6$ on $[0,t_1],$ where $c_5$ and $c_6$ depend on $K_0,$
\end{lemma}
\begin{proof}
In view of Lemmas \ref{lem: lower} and \ref{lem: upper}, it suffices to show that $\|W\|$ remains bounded on $[0,t_1]$. Consider the test function
\[\Theta=\frac 12\log(\|W\|^2)-\alpha\log h.\]
Assume, without loss of generality, that $c_1> 1;$ otherwise we replace $h$ by $2h/c_1.$
Using the parabolic maximum principle we show that for some $\alpha$ large enough $\Theta(\cdot,t)$ is always negative on $[0,t_1]$. If the conclusion of the theorem is false, we may choose $(x_0,t_0)$ with $t_0>0$ and such that $\Theta(x_0,t_0)=0$, $\Theta(x, t_0) \leq 0$, and $\Theta(x,t) < 0$ for $t<t_0$. Then,
\begin{align*}
0 \leq& \dot{\Theta}-\psi\frac{\mathcal{K}^{kl}}{\mathcal{K}^2}\Theta_{;kl} \\
&= -\frac{\psi}{\|W\|^2} \frac{\mathcal{K}^{kl}}{\mathcal{K}^2} w_{i;k}^j w_{j;l}^i + \frac{2\psi}{\|W\|^4} \frac{\mathcal{K}^{kl}}{\mathcal{K}^2} w^j_iw^s_r w^i_{j;k} w^r_{s;l} \\
&+ \psi \frac{\mathcal{K}^{kl}}{\mathcal{K}^2} w_{kr}w_l^r - \psi(n+1) \frac{(w^2)_i^jw_j^i}{\mathcal{K}\|W\|^2} \\
&+ \frac{\psi w^i_j}{\|W\|^2} \left(\frac{\mathcal{K}^{kl,rs}}{\mathcal{K}^2} w_{kl;i} g^{jp}jw_{rs;p} - 2\frac{g^{jp} \mathcal{K}_{;i}\mathcal{K}_{;p}}{\mathcal{K}^3}\right) \\
&+ \left(\frac{2}{\mathcal{K}^2} g^{jp}\psi_{;i} \mathcal{K}_{;j} - \frac{1}{\mathcal{K}} g^{jp} \psi_{;ip}\right) \frac{w^i_j}{\|W\|^2} \\
& +(n-1) \frac{\alpha\psi}{h\mathcal{K}} + \frac{\alpha}{h\mathcal{K}} (F\cdot\nabla\psi) - \frac{\alpha\psi}{h^2} \frac{\mathcal{K}^{kl}}{\mathcal{K}^2} h_{;k}h_{;l} - \alpha\psi \frac{\mathcal{K}^{kl}}{\mathcal{K}^2} w_{kr}w_l^r.
\end{align*}
Working in an orthonormal frame at $(x_0,t_0)$, with respect to which we may write $$\mathcal{K}^{kl,rs}w_{kl;i}w_{rs;i}=\mathcal{K}^{kk,ll}w_{kk;i}w_{ll;i}-\mathcal{K}^{kk,ll}w_{kl;i}^2,$$ we obtain
\begin{align*}
0 \leq& -\frac{\psi}{\|W\|^2} \mathcal{K}^{ii} \sum_lw_{ll;i}^2 - \frac{\psi}{\|W\|^2} \mathcal{K}^{ii} \sum_{p\ne q}w_{pq;i}^2 + \frac{2\psi}{\|W\|^4} \mathcal{K}^{ii} \left(\sum_j\kappa_jw_{jj;i}\right)^2 \\
&+ \psi\mathcal{K}^{ii} w_{ii}^2 - \psi(n+1) \mathcal{K} \sum_i\frac{w_{ii}^3}{\|W\|^2} \\
&+ \frac{\psi w_{ii}}{\|W\|^2} \left(\mathcal{K}^{pp,qq}w_{pp;i}w_{qq;i} - \mathcal{K}^{pp,qq}w_{pq;i}^2 - 2\frac{(\mathcal{K}_{;i})^2}{\mathcal{K}}\right) \\
&+ (2\psi_{;i}\mathcal{K}_{;i} - \mathcal{K}\psi_{;ii}) \frac{w_{ii}}{\|W\|^2} \\
&+ (n-1) \frac{\alpha\psi\mathcal{K}}{h} + \frac{\alpha\mathcal{K}}{h}(F\cdot\nabla\psi) - \frac{\alpha\psi}{h^2}\mathcal{K}^{kl} h_{;k} h_{;l} - \alpha\psi\mathcal{K}^{ii}w_{ii}^2.
\end{align*}
From the vanishing of the first variation, at $(x_0,t_0)$ we have
\begin{equation}\label{eq: max}
0 = \Theta_{;k} = \frac{\kappa_iw_{ii;k}}{\|W\|^2} - \alpha\frac{h_{;k}}{h},
\end{equation}
We may assume at $x_0$ that $w_1^1=\kappa_1=\max\{\kappa_i:1\leq i\leq n\}.$ Therefore,
\begin{equation}\label{eq: max1}
\Theta(x_0,t_0)=0\Rightarrow\frac{c_1^{\alpha}}{\sqrt{n}}\leq \kappa_1\leq c_2^{\alpha}.
\end{equation}
On the other hand, since $\psi$ is bounded below in view of the bounds on $h$ in the hypotheses of the theorem,
\begin{align}\label{eq: 1}
\psi_{;i} \leq C_0\kappa_i \Rightarrow 2\psi_{;i} \mathcal{K}_{;i}&\leq \frac{\varepsilon\psi}{c_4} (\mathcal{K}_{;i})^2 + \frac{c_4C_0^2}{\psi\varepsilon}\kappa_i^2\nonumber\\
&\leq \frac{\varepsilon\psi}{c_4} (\mathcal{K}_{;i})^2 + C\frac{\psi\kappa_i^2}{\varepsilon}\nonumber\\
&\leq \varepsilon\psi \frac{(\mathcal{K}_{;i})^2}{\mathcal{K}} + C(\epsilon) \psi\kappa_i^2,
\end{align}
where $c_4$ is from Lemma \ref{lem: lower},
and
\begin{equation}\label{eq: 2}
\psi_{;ii} \geq - C - C\kappa_i - C\kappa_i^2 + w_{ii;k} d_{\nu} \psi(\partial_k).
\end{equation}
Using (\ref{eq: max}) in (\ref{eq: 2}) we obtain
\begin{align}\label{eq: 3}
-\frac{\mathcal{K}}{\|W\|^2}\sum_i\kappa_i \psi_{;ii}&\leq \frac{\mathcal{K}}{\|W\|^2}\sum_i \kappa_i(C+C\kappa_i+C\kappa_i^2-w_{ii;k}d_{\nu}\psi(\partial_k))\nonumber\\
&\leq \frac{\mathcal{K}}{\|W\|^2}\sum_i \kappa_i(C+C\kappa_i+C\kappa_i^2)-\frac{\alpha\mathcal{K}}{h} \sum_k h_{;k} d_{\nu}\psi(\partial_k)\nonumber\\
&=\frac{\mathcal{K}}{\|W\|^2}\sum_i \kappa_i(C+C\kappa_i+C\kappa_i^2)-\frac{\alpha\mathcal{K}}{h}\sum_i \kappa_i(\partial_i\cdot F) d_{\nu}\psi(\partial_i)\nonumber\\
&\leq \frac{\psi}{\|W\|^2}\sum_i \kappa_i(C+C\kappa_i^2)-\frac{\alpha\mathcal{K}}{h}\sum_i \kappa_i (\partial_i \cdot F) d_{\nu}\psi(\partial_i).
\end{align}
To get the last inequality, we used that $\mathcal{K}$ is bounded above and $\psi$ is bounded below.

Combining  (\ref{eq: 1}) and (\ref{eq: 3}) implies that (where the constant $C$ depends on $\varepsilon$)
\begin{align}\begin{split}\label{eq:L10-1}
0\leq& -\frac{\psi}{\|W\|^2}\mathcal{K}^{ii}\sum_lw_{ll;i}^2-\frac{\psi}{\|W\|^2}\mathcal{K}^{ii}\sum_{p\ne q}w_{pq;i}^2
+\frac{2\psi}{\|W\|^4}\mathcal{K}^{ii}\left(\sum_j\kappa_jw_{jj;i}\right)^2\\
&+\psi\mathcal{K}^{ii}w_{ii}^2-\psi(n+1)\mathcal{K}\sum_i\frac{w_{ii}^3}{\|W\|^2}\\
&+\frac{\psi w_{ll}}{\|W\|^2}\left(\mathcal{K}^{pp,qq}w_{pp;l}w_{qq;l}-\mathcal{K}^{pp,qq}w_{pq;l}^2
-(2-\varepsilon)\frac{(\mathcal{K}_{;l})^2}{\mathcal{K}}\right)\\
&+\frac{\psi}{\|W\|^2}\sum_i \kappa_i(C+C\kappa_i^2)-\frac{\alpha\mathcal{K}}{h}\sum_i \kappa_i(\partial_i\cdot F) d_{\nu}\psi(\partial_i)\\
&+(n-1)\frac{\alpha\psi\mathcal{K}}{h}+\frac{\alpha\mathcal{K}}{h}\sum_s(\partial_s\cdot F) d_F\psi(\partial_s)+\frac{\alpha\mathcal{K}}{h}\sum_i\kappa_i(\partial_i\cdot F) d_{\nu}\psi(\partial_i)\\
&-\frac{\alpha\psi}{h^2}\mathcal{K}^{ii}w_{ii}^2(\partial_i\cdot F)^2 -\alpha\psi\mathcal{K}^{ii}w_{ii}^2\\
\leq&\frac{\psi}{\|W\|^2}\left(\sum_l\kappa_l\left(C+C\kappa_l^2+\frac{(\beta
-2+\varepsilon)(\mathcal{K}_{;l})^2}{\mathcal{K}}\right)-n\mathcal{K}\sum_l\kappa_l^3+\mathcal{K}^{ii}w_{ii}^2\|W\|^2\right)\\
&+\alpha\psi\left(\frac{n\mathcal{K}}{h}-\mathcal{K}^{ii}w_{ii}^2-\frac{\mathcal{K}^{ii}w_{ii}^2(\partial_i\cdot F)^2}{h^2}+\frac{\mathcal{K}}{h\psi}\sum_s(\partial_s\cdot F) d_F\psi(\partial_s)\right)\\
&-\psi\sum_i\left(A_i+B_i+C_i+D_i-E_i\right)
-\frac{\alpha\psi\mathcal{K}}{h}-\psi\mathcal{K}\sum_i\frac{w_{ii}^3}{\|W\|^2},
\end{split}\end{align}
where $\beta \in \mathbb{R}$ is any constant, and
\[A_i=\frac{\beta}{\|W\|^2\mathcal{K}} w_{ii}(\mathcal{K}_{;i})^2-\frac{w_{ii}}{\|W\|^2}\sum_{p,q}\mathcal{K}^{pp,qq}w_{pp;i}w_{qq;i},\]

\[B_i=\frac{2}{\|W\|^2}\sum_{j}w_{jj}\mathcal{K}^{jj,ii}w^2_{jj;i},\quad C_i=\frac{2}{\|W\|^2}\sum_{j\neq i}\mathcal{K}^{jj}w_{jj;i}^2,\]

\[D_i=\frac{1}{\|W\|^2}\mathcal{K}^{ii}\sum_j w_{jj;i}^2,\quad
E_i=\frac{2}{\|W\|^4}\mathcal{K}^{ii}\left(\sum_j w_{jj}w_{jj;i}\right)^2.\]

The terms $B_i$ and $C_i$ deserve some explanation. $C_i$ comes from the second term in \eqref{eq:L10-1}, which reads
\[-\frac{\psi}{\|W\|^2}\sum_i\mathcal{K}^{ii}\sum_{p\neq q}w_{pq;i}^2\leq -\frac{\psi}{\|W\|^2}\sum_{p\neq q}\mathcal{K}^{pp}w_{pq;p}^2-\frac{\psi}{\|W\|^2}\sum_{p\neq q}\mathcal{K}^{qq}w_{pq;q}^2, \]
which is exactly $C_i$ due to the Codazzi equation.

The third line of \eqref{eq:L10-1} arises from
\begin{align*}
\mathcal{K}^{kl,rs}w_{kl;i}w_{rs;j}w^{ij}=\sum_i w_{ii}\left(\sum_{p,q}\frac{\partial^2\mathcal{K}}{\partial \kappa_p\partial\kappa_q}w_{pp;i}w_{qq;i}+\sum_{p\neq q}\frac{\frac{\partial\mathcal{K}}{\partial \kappa_p}-\frac{\partial\mathcal{K}}{\partial\kappa_q}}{\kappa_p-\kappa_q} w_{pq;i}^2\right);
\end{align*}
see, for example, \cite[Lemma~2.1.14]{Gerhardt:/2006} for this relation. Since the second term in the bracket is negative and the hypersurface is convex, we can proceed in the same way as we derived $C_i$ and just throw away all indices $i$ which are neither $p$ or $q$. This gives term $B_i$. The first term in the big bracket goes into $A_i$.


In Corollary \ref{Alternative} of the appendix we will present an adaption of the method developed in \cite{Guan} to deal with the curvature derivative terms $A_i,B_i,C_i,D_i,E_i$. There we prove that by taking $\beta=2-\varepsilon$ with $0<\varepsilon<1$ we obtain the following alternative: Either there exist positive numbers $\delta_1,\dots,\delta_n$, which only depend on the dimension, such that
\[\kappa_i\geq \delta_i\kappa_1\quad\forall 1\leq i\leq n\]
or
\[A_i+B_i+C_i+D_i-E_i\geq 0\quad\forall 1\leq i\leq n.\]
By taking $\alpha$ large in \eqref{eq: max1}, in the first case we get a contradiction to the bound on the Gauss curvature. In the second case, using also $\mathcal{K}^{ii}w_{ii}^2 = \mathcal{K} \sum_i \kappa_i$, \eqref{eq:L10-1} yields


\begin{align*}
0 \leq& \frac{\psi}{\|W\|^2}\left(\sum_l \kappa_l(C + C \kappa_l^2) - n \mathcal{K} \sum_l \kappa_l^3\right)- (\alpha-1) \mathcal{K} \psi \sum_i \kappa_i \\
& + \psi \left((n-1)\frac{\mathcal{K}}{h} - \frac{\mathcal{K}}{h^2} \sum \kappa_i (\partial_i \cdot F)^2 + \frac{\mathcal{K}}{h\psi} \sum_l  (\partial_l \cdot F) d_F \psi (\partial_l) \right) \alpha
\end{align*}
so
$0\leq \frac{C(\varepsilon)\kappa_1^3}{\|W\|^2} - (\alpha-1) \mathcal{K}\psi \kappa_1 + C \alpha,$
where we discarded $-(\alpha-1)\mathcal{K}\psi \sum_{i\neq 1} \kappa_i \leq 0$ and used the bounds on $h, \psi$ and $\mathcal{K}$ to bound $\kappa_1$ in terms of $\kappa_1^3$.

Now take $\alpha$ such that $(\alpha-1) \mathcal{K}\psi \geq C(\varepsilon)+1$. Therefore, in view of (\ref{eq: max1})
\begin{equation}\label{x}
\begin{split}
0 &\leq \frac{C(\varepsilon)\kappa_1^3}{\|W\|^2} - (\alpha-1) \mathcal{K}\psi\kappa_1 + C\alpha \\
&\leq C(\epsilon)\left(\frac{\kappa_1^2}{\|W\|^2} - 1\right) \kappa_1 - \kappa_1 + C\alpha \\
&\leq -\frac{c_1^{\alpha}}{\sqrt{n}} + C\alpha.
\end{split}
\end{equation}
Taking $\alpha$ large enough yields a contradiction.
\end{proof}
%\begin{remark}
%In $\mathbb{R}^3,$ to deduce upper and lower bounds on principal curvatures we could instead apply the maximum principle to a test function considered by Spruck-Xiao \cite{Spruck} for the prescribed scalar curvature problem:
%\begin{align*}
%\Theta:=\log \zeta -\log(h-a)+\alpha\|F\|^2
%\end{align*}
%where $\zeta=\sup\{w_{ij}\eta^i\eta^j:\|\eta\|=1\}$ and $0<a\leq c_1/2;$ see Section \ref{appendix}.
%\end{remark}
\begin{proposition}\label{prop: expansion to infty}
The solution to (\ref{e: flow0}) satisfies $\lim\limits_{t\to T}\max h_{K_t}=\infty.$
\end{proposition}
\begin{proof}
First, let $p> n+1.$ In this case, by comparing with outer balls, the flow exists on $[0,\infty).$
Consider an origin centered ball $B_r$, such that $K_0\supseteq B_r.$ Then $K_t\supseteq B_{r(t)},$ where $$r(t)=\left((\min h_{K_0})^{p-n-1}+t(p-n-1)\min \varphi \right)^{\frac{1}{p-n-1}}$$ and $B_{r(t)}$ expands to infinity as $t$ approaches $\infty$. When $p=n+1$ the argument is similar. Second, if $p<n+1$, then the flow exists only on a finite time interval. If $\max h_{K_t}<\infty$, then by Lemmas \ref{lem: lower}, \ref{lem: upper} and \ref{lem: final}, the evolution equation (\ref{e: flow0}) is uniformly parabolic on $[0,T)$. Thus, the result of Krylov and Safonov \cite{Krylov-Safonov} and standard parabolic theory allow us to extend the solution smoothly past time $T$, contradicting its maximality.
\end{proof}
\section{convergence of normalized solutions}\label{sec: nor conv}
\subsection{Convergence in $C^{1}$, $p\neq 1>-n-1$}\label{sec}
By the proof of \cite[Corollary 7.5]{Ivaki 2014-gauss}, there exist $r,R$ such that
\begin{align}\label{ratio}
0<r\leq h_{\tilde{K}_t}\leq R<\infty.
\end{align}
Therefore, a subsequence of $\{\tilde{K}_{t_k}\}$ converges in the Hausdorff distance to a limiting shape $\tilde{K}_{\infty}$ with the origin in its interior. The argument of \cite[Section 8.1]{Ivaki 2014-gauss} implies  $$\varphi h_{\tilde{K}_{\infty}}^{1-p}f_{\tilde{K}_{\infty}}=c,$$ where $f_{\tilde{K}_{\infty}}$ is the positive continuous curvature function of $\tilde{K}_{\infty}$ and $c$ is some positive constant. By \cite[Fact 8.1]{Ivaki 2014-gauss}, $\tilde{K}_{\infty}$  is smooth and strictly convex. The $C^1$-convergence follows, which is purely geometric and does not depend on the evolution equation, from \cite[Lemma 13]{Andrews 1997}.
\begin{remark}\label{rem}
Section \ref{sec} completes the discussion on the existence of solutions to the smooth, even $L_p$-Minkowski problems in $\mathbb{R}^{n+1}$ for $p\neq 1>-n-1.$ The next section discusses  the $C^{\infty}$ convergence when $p\neq1\leq n+1$, and also when $p=1$ and solutions are origin-symmetric.
\end{remark}

\subsection{Convergence in $C^{\infty}$}

By \cite[Lemma 9.2]{Ivaki 2014-gauss}, there is a uniform upper bound on the Gauss curvature of the normalized solution when $p\leq n+1.$
In the following, we first obtain a uniform lower bound on the Gauss curvature of the normalized solution $\tilde{K}_t$.


Let $h:\mathbb{S}^{n}\times[0,T)\to\mathbb{R}^{n+1}$ be a solution of equation (\ref{eq: flow4}). Then for each $\lambda >0$, $\bar{h}$ defined by
\begin{align*}
\bar{h}&:\mathbb{S}^{n}\times\left[0,T/\lambda^{\frac{1+n-p}{n+1}}\right)\to\mathbb{R}^{n+1}\\
\bar{h}(u,t)&=\lambda^{\frac{1}{n+1}} h\left(u,\lambda^{\frac{1+n-p}{n+1}}t\right)
\end{align*}
is also a solution of evolution equation (\ref{eq: flow4}) but with the initial data $\lambda^{\frac{1}{n+1}} h\left(\cdot,0\right).$

For each \emph{fixed} time $t\in[0,T),$ define $\bar{h}$ a solution of (\ref{eq: flow4}), by the rescaling property, as follows
\begin{align*}
\bar{h}(u,\tau)=\left(\frac{V(B)}{V(K_t)}\right)^{\frac{1}{n+1}}h\left(u,
t+\left(\frac{V(B)}{V(K_t)}\right)^{\frac{1+n-p}{n+1}}\tau\right).
\end{align*}
Note that $\bar{h}(\cdot,0)$ is the support function of $\left(V(B)/V(K_t)\right)^{\frac{1}{n+1}}K_t$; therefore,
\[r\leq \bar{h}(u,0)\leq R.\]
Write $\bar{K}_{\tau}$ for the convex body associated with $\bar{h}(\cdot,\tau)$ and let $B_c$ denote the ball of radius $c$ centered at the origin. Since $B_{R}$ encloses
$\bar{K}_0,$ the comparison principle implies that $B_{2R}$
will enclose $\bar{K}_{\tau}$ for $\tau\in[0,\delta],$
where $\delta$ depends only on $p,R,\psi$. By the first statement of Lemma \ref{lem: upper} applied to $\bar{h}$, there is a uniform lower bound (depending only on $r,R,p,\psi$) on the Gauss curvature of $\bar{K}_{\frac{\delta}{2}}.$

On the other hand, the volume of $\bar{K}_{\frac{\delta}{2}}$ is bounded above by $V(B_{2R});$ therefore,
\[\displaystyle\frac{V(B)}{V(B_{2R})}\leq c_t:=\frac{V(K_t)}{V\left(K_{t+\left(\frac{V(B)}{V(K_t)}\right)^{\frac{1+n-p}{n+1}}\frac{\delta}{2}}\right)}\leq 1\]
for all $t\in [0,T)$. Consequently,
\begin{align*}
\left(\frac{V(B)}{V\left(K_{t+\left(\frac{V(B)}{V(K_t)}\right)^{\frac{1+n-p}{n+1}}\frac{\delta}{2}}\right)}\right)^{\frac{1}{n+1}}h\left(u,
t+\left(\frac{V(B)}{V(K_t)}\right)^{\frac{1+n-p}{n+1}}\frac{\delta}{2}\right)
=c_t^{\frac{1}{n+1}}\cdot\bar{h}(\cdot,\frac{\delta}{2})
\end{align*}
has Gauss curvature bounded below for all $t\in [0,T)$.

Now we show that for every
 $\tilde{t}\in\left[\left(V(B)/V(K_0)\right)^{\frac{1+n-p}{n+1}}\frac{\delta}{2},T\right)$, we can find $t\in[0,T)$ such that
\[\tilde{t}=t+\left(\frac{V(B)}{V(K_t)}\right)^{\frac{1+n-p}{n+1}}\frac{\delta}{2}.\]
Define $f(t)=t+\left(\frac{V(B)}{V(K_t)}\right)^{\frac{1+n-p}{n+1}}\frac{\delta}{2}-\tilde{t}$ on $[0,T)$.
$f$ is continuous, and
\[\left\{
    \begin{array}{ll}
      f(T)=T-\tilde{t}>0, & p<n+1 \\
      f(\infty)=\infty, & p= n+1 \\
      f(0)\leq 0& p\leq  n+1 .
    \end{array}
  \right.
\]
The claim follows.

Next we obtain uniform lower and upper bounds on the principal curvatures of the normalized solution.

Consider the convex bodies $\tilde{K}_{\tau}:=\left(\frac{V(B)}{V(K_t)}\right)^{\frac{1}{n+1}}K_t,$ where
$$\tau(t):=\int_0^t \left(\frac{V(K_t)}{V(B)}\right)^{\frac{1+n-p}{n+1}}dt,\footnote{Suppose $p<n+1.$ For each $t\in [0,T)$ by the comparison principle we have
\[\frac{(\max h_{K_t})^{p-n-1}}{(n+1-p)\max \varphi}\leq T-t\leq \frac{(\min h_{K_t})^{{p-n-1}}}{(n+1-p)\min \varphi}.\]
Therefore, since $\frac{\max h_{K_t}}{\min h_{K_t}}\leq \frac{R}{r}$ (see (\ref{ratio})), we get
\begin{align*}
c_1(T-t)^{\frac{1}{p-n-1}}\leq \min h_{K_t} \leq \left(\frac{V(K_t)}{V(B)}\right)^{\frac{1}{n+1}}\leq \max h_{K_t}\leq c_2(T-t)^{\frac{1}{p-n-1}}.
\end{align*}
Thus $\lim_{t\to T}\tau(t)=\infty.$}$$
Let us furnish all geometric quantities associated with $\tilde{K}_{\tau}$ by an over-tilde.
The evolution equation of $\tilde{h}_{\tau}$ is given by
\[\partial_\tau \tilde{h}_{\tau}=\varphi \tilde{h}^{2-p}\tilde{S}_n-\frac{\int_{\mathbb{S}^{n}}\varphi \tilde{h}^{2-p}\tilde{S}_n^2d\sigma}{(n+1)V(B)}\tilde{h}.\]
Since $\frac{\int_{\mathbb{S}^{n}}\varphi \tilde{h}^{2-p}\tilde{S}_n^2d\sigma}{(n+1)V(B)}$ is uniformly bounded above,
applying the maximum principle to
$\Theta=\frac 12\log(\|\tilde{W}\|^2)-\alpha\log \tilde{h},$
and arguing as in the proof of Lemma \ref{lem: final}, we see that $\|\tilde{W}\|$ has a uniform upper bound. This in turn, in view of our lower and upper bounds on the Gauss curvature of $\tilde{K}_{\tau}$, implies that we have uniform lower and upper bounds on the principal curvatures of $\tilde{K}_{\tau}$. Higher order regularity estimates and convergence in $C^{\infty}$ for a subsequence of $\{\tilde{K}_{\tau}\}$ follow from Krylov-Safonov \cite{Krylov-Safonov}, standard parabolic theory and the Arzel\`{a}-Ascoli theorem. The convergence for the full sequence when $p\geq 1$ follows from the uniqueness of the self-similar solutions to (\ref{def: self similar}); see \cite{Lu1,39}.

\section{Appendix}\label{appendix}

\subsection*{Evolution of polar bodies}

Let $K$ be a smooth, strictly convex body with the origin in its interior. Suppose $\partial K$, the boundary of $K,$ is parameterized with the radial function $r.$
The metric $[g_{ij}]_{1\leq i,j\leq n-1}$, unit normal $\nu$, support function $h$, and the second fundamental form $[w_{ij}]_{1\leq i,j\leq n-1}$ of $\partial K$
can be written in terms of $r$ and its partial derivatives as follows:
\begin{description}
  \item[a] $ g_{ij}=r^2\bar{g}_{ij}+\bar{\nabla}_ir\bar{\nabla}_jr,$
  \item[b] $ \nu=\frac{r z-\bar{\nabla}r}{\sqrt{r^2+\|\bar{\nabla}r\|^2}},$
  \item[c] $ h=\frac{r^2}{\sqrt{r^2+\|\bar{\nabla}r\|^2}},$
  \item[d] $ w_{ij}=\frac{-r\bar{\nabla}_i\bar{\nabla}_jr+2\bar{\nabla}_ir\bar{\nabla}_jr+
  r^2\bar{g}_{ij}}{\sqrt{r^2+\|\bar{\nabla}r\|^2}}.$
\end{description}
Since $\frac{1}{r}$ is the support function of $K^{\ast}$, we can calculate the entries of $[\mathfrak{r}^{\ast}_{ij}]_{1\leq i,j\leq n-1}$:
\[\mathfrak{r}^{\ast}_{ij}=\bar{\nabla}_i\bar{\nabla}_j\frac{1}{r}+\frac{1}{r}\bar{g}_{ij}=
\frac{-r\bar{\nabla}^2_{ij}r+2\bar{\nabla}_ir\bar{\nabla}_jr+r^2\bar{g}_{ij}}{r^3}.\]
Thus, using (\textbf{d}) we get
\begin{align*}
\mathfrak{r}^{\ast}_{ij}=\frac{\sqrt{r^2+\|\bar{\nabla}r\|^2}}{r^3}w_{ij}.
\end{align*}
\begin{lemma}\label{app1}
As $K_t$ evolve by (\ref{eq: flow4}), their polars $K_t^{\ast}$ evolve as follows:
\[\partial_th^{\ast}=-\varphi\left(\frac{h^{\ast}u+\bar{\nabla} h^{\ast}}{\sqrt{h^{\ast2}+|\bar{\nabla} h^{\ast}|^2}}\right)\frac{(h^{\ast2}+|\bar{\nabla} h^{\ast}|^2)^{\frac{n+1+p}{2}}}{h^{\ast n+1}S_n^{\ast}},~~h^{\ast}(\cdot,t):=h_{K_t^{\ast}}(\cdot).\]
\end{lemma}
\begin{proof}
To obtain the evolution equation of $h_{K_t^{\ast}}$, we first need to parameterize $M_t$ over the unit sphere
\[F=r(z(\cdot,t),t)z(\cdot,t):\mathbb{S}^{n}\to\mathbb{R}^{n+1},\]
where $r(z(\cdot,t),t)$ is the radial function of $M_t$ in the direction $z(\cdot,t).$
Note that
\[\partial_t r=\varphi \frac{h^{2-p}}{\mathcal{K}}\frac{\sqrt{r^2+\|\bar{\nabla}r\|^2}}{r},\]
and
\[\mathcal{K}=\frac{\det w_{ij}}{\det g_{ij}},\quad \frac{1}{S_n^{\ast}}=\frac{\det \bar{g}_{ij}}{\det \mathfrak{r}_{ij}^{\ast}},\quad \frac{\det \bar{g}_{ij}}{\det g_{ij}}=\frac{1}{r^{2n-2}(r^2+\|\bar{\nabla}r\|^2)},
\quad h=\frac{1}{\sqrt{h^{\ast 2}+\|\bar{\nabla}h^{\ast}\|^2}}.\]
Now we calculate
\begin{align*}
\partial_t h^{\ast}&=\partial_t \frac{1}{r}\\
&=- \frac{h^{2-p}}{\mathcal{K}}\frac{\sqrt{r^2+\|\bar{\nabla}r\|^2}}{r^3}\varphi\circ\nu\\
&=-h^{2-p}\frac{\sqrt{r^2+\|\bar{\nabla}r\|^2}}{r^3}\frac{\det g_{ij}}{\det w_{ij}}\varphi\circ\nu\\
&=-h^{2-p}\frac{\sqrt{r^2+\|\bar{\nabla}r\|^2}}{r^3}\frac{\det \bar{g}_{ij}}{\det \mathfrak{r}_{ij}^{\ast}}\frac{\det g_{ij}}{\det \bar{g}_{ij}}\frac{\det \mathfrak{r}_{ij}^{\ast}}{\det w_{ij}}\varphi\circ\nu\\
&=-\left(\frac{\sqrt{r^2+\|\bar{\nabla}r\|^2}}{r^3}\right)^{n+1}\frac{r^{2n-2}(r^2+\|\bar{\nabla}r\|^2)}{(h^{\ast 2}+\|\bar{\nabla}h^{\ast}\|^2)^{\frac{2-p}{2}}}\frac{\varphi\circ\nu}{S_n^{\ast}}.
\end{align*}
Replacing $r$ by $1/h^{\ast}$ and taking into account (\textbf{b}) finishes the proof.
\end{proof}

\subsection*{Estimates for curvature derivatives}
For convenience we present some of the main ideas, how one can prove the alternative in Lemma \ref{lem: final} about balancing the curvature derivatives. This method was used in \cite{Guan} for a similar stationary prescribed curvature equation.
In the following $\sigma_k$ denotes the $k$-th elementary symmetric function of principal curvatures. We begin by recalling the following special case of inequality (2.4) from \cite[Lemma 2.2]{Guan}, which can be deduced easily by differentiating
\[\log G=\log \left(\frac{\sigma_k}{\sigma_l}\right)^{\frac{1}{k-l}}\] twice, using the concavity of $G$ and applying the Schwartz inequality.
For any $\delta>0$ and $1\leq l<n$ we have
\begin{align*}
-&\mathcal{K}^{pp,qq}w_{pp;i}w_{qq;i}+\left(1-\frac{1}{n-l}+\frac{1}{(n-l)\delta}\right)\frac{(\mathcal{K}_{;i})^2}{\mathcal{K}}\geq \\
&\left(1+\frac{1-\delta}{n-l}\right)\frac{\mathcal{K}((\sigma_l)_{;i})^2}{\sigma_l^2}-\frac{\mathcal{K}}{\sigma_l}\sigma_l^{pp,qq}w_{pp;i}w_{qq;i}.
\end{align*}
In particular, by taking $\delta=\frac{1}{2-\varepsilon}$, we have
\begin{align}\begin{split}\label{key ineq}
(2-\varepsilon)\frac{(\mathcal{K}_{;i})^2}{\mathcal{K}}-\mathcal{K}^{pp,qq}w_{pp;i}w_{qq;i}&\geq \left[1+\frac{1-\varepsilon}{(n-1)(2-\varepsilon)}\right]\frac{\mathcal{K}((\sigma_l)_{;i})^2}{\sigma_l^2}\\
					&\hphantom{=}-\frac{\mathcal{K}\sigma_l^{pp,qq}w_{pp;i}w_{qq;i}}{\sigma_l}.
\end{split}\end{align}



\begin{remark}
Note that the term $A_i$ looks slightly different from the term $A_i$ in \cite{Guan}, where the $\mathcal{K}$ is not present in the denominator. We have to define $A_i$ in the way we did, because due to the inverse nature of the curvature flow equation we obtain an extra good derivative term, into which we can absorb the $\beta$-part of $A_i$, if it is not too large. Fortunately the proofs of \cite[Lemma~4.2, Lemma~4.3]{Guan} also work for some $\beta<2$ as we will indicate shortly.
\end{remark}

\begin{lemma}\label{app2}
For each $i\ne 1,$ if $\sqrt{3}\kappa_i\leq \kappa_1,$ we have
\[A_i+B_i+C_i+D_i-E_i\geq 0.\]
\end{lemma}
\begin{proof}
Note that from (\ref{key ineq}) it follows that $A_i\geq 0.$ The proof of that $B_i+C_i+D_i-E_i\geq 0$ can literally be taken from \cite[Lemma 4.2]{Guan}, starting with \cite[Equ.~(4.10)]{Guan}.
\end{proof}

In the following proof we will write $\sigma_n=\mathcal{K}$ for a better comparability with \cite[Lemma~4.3]{Guan}. Also denote by $\sigma_k(\kappa|i)$ the $k$-th elementary symmetric polynomial in the variables $\kappa_1,\dots,\kappa_{i-1},\kappa_{i+1},\dots,\kappa_n$ and $\sigma_k(\kappa|ij)$ accordingly.


\begin{lemma}\label{app3}
For $\lambda=1,\cdots,n-1$ suppose there exists some $\delta\leq 1$ such that $\kappa_{\lambda}/\kappa_1\geq \delta.$ There exists a sufficiently small positive constant $\delta'$ depending on $\delta$, $\epsilon$ and the bounds for $\mathcal{K}$, such that if $\kappa_{\lambda+1}/\kappa_1\leq \delta',$ we have
\[A_i+B_i+C_i+D_i-E_i\geq 0\quad \text{for}~i=1,\cdots,n.\]
\end{lemma}
\begin{proof}
This corresponds to \cite[Lemma~4.3]{Guan}. We highlight the main estimates in this proof. First of all, from the estimates in \cite[Lemma~4.2]{Guan} one can extract the following estimate:
\begin{align}\begin{split}\label{app3-1}
\|W\|^4(B_i+C_i+D_i-E_i)&\geq \|W\|^2\sum_{j\neq i} \left(\sigma_{n-1}(\kappa|j)-2\sigma_{n-1}(\kappa|ij)\right)w_{jj;i}^2\\
				&\hphantom{=}-w_{ii}^2\sigma_{n}^{ii}w_{ii;i}^2,
\end{split}\end{align}
compare \cite[Equ.~(4.16), (4.17)]{Guan}.
Now we need to estimate $A_i$, where we recall taking $\beta=2-\varepsilon$ with $0<\varepsilon<1$. From \eqref{key ineq} we get for all $1\leq \lambda<n$ and for all $1\leq i\leq n\colon$
\begin{align}\begin{split}\label{}
A_i&=\frac{\beta w_{ii}}{\|W\|^2\sigma_n}((\sigma_n)_{;i})^2-\frac{w_{ii}}{\|W\|^2}\sum_{p,q}\sigma_n^{pp,qq}w_{pp;i}w_{qq;i}\\
		&\geq \frac{w_{ii}}{\|W\|^2}\left(1+\frac{1-\varepsilon}{(n-1)(2-\varepsilon)}\right)\frac{\sigma_n((\sigma_{\lambda})_{;i})^2}{\sigma_{\lambda}^2}\\
					&\hphantom{=}-\frac{w_{ii}}{\|W\|^2}\frac{\sigma_n\sum_{p,q}\sigma_{\lambda}^{pp,qq}w_{pp;i}w_{qq;i}}{\sigma_{\lambda}}\\
                    &=\frac{w_{ii}\sigma_n}{\|W\|^2\sigma_{\lambda}^2}\Big[\left(1+\frac{1-\varepsilon}{(n-1)(2-\varepsilon)}\right)\sum_{a}\left(\sigma_{\lambda}^{aa}w_{aa;i}\right)^2\\
                    	&\hphantom{\frac{w_{ii}\sigma_n}{\|W\|^2\sigma_{\lambda}^2}\Big(}+\frac{1-\varepsilon}{(n-1)(2-\varepsilon)}\sum_{a\neq b}\sigma_\lambda^{aa}\sigma_{\lambda}^{bb}w_{aa;i}w_{bb;i}\\
                        &\hphantom{\frac{w_{ii}\sigma_n}{\|W\|^2\sigma_{\lambda}^2}\Big(}+\sum_{a\neq b}\left(\sigma_{\lambda}^{aa}\sigma_{\lambda}^{bb}-\sigma_{\lambda}\sigma_{\lambda}^{aa,bb}\right)w_{aa;i}w_{bb;i}\Big].
\end{split}\end{align}

For sufficiently small $\delta'$ and $\lambda=1$ the simple estimates \cite[Equ.~(4.19), (4.20)]{Guan} give
\begin{equation}
\|W\|^4A_i\geq w_{ii}^2\sigma_n^{ii}w_{11;i}-C_{\epsilon}w_{ii}\sum_{a\neq 1}w_{aa;i}^2.
\end{equation}

Combining this with \eqref{app3-1} for $i=1$ yields, also using $\sigma_{n-1}(\kappa|ij)=0,$
\begin{align}\begin{split}\label{app3-2}
\|W\|^2(A_1+B_1+C_1+D_1-E_1)&\geq \sum_{j\neq 1} \sigma_{n-1}(\kappa|j)w_{jj;1}^2-\frac{C_{\epsilon}}{w_{11}}\sum_{j\neq 1}w_{jj;1}^2\\
		&=\sum_{j\neq 1}\left(\frac{\sigma_n}{w_{jj}}-\frac{C_{\epsilon}}{w_{11}}\right)w_{jj;1}^2\\
        &\geq \sum_{j\neq 1}\left(\frac{\sigma_n}{\delta' w_{11}}-\frac{C_{\epsilon}}{w_{11}}\right)w_{jj;1}^2,
\end{split}\end{align}
which is non-negative for $\delta'$ sufficiently small. Hence the lemma is true in the case $\lambda=1.$

For $\lambda>1$ the series of elementary estimates \cite[Equ.~(4.22)-(4.27)]{Guan} gives
\[
\|W\|^4A_i\geq w_{ii}^2\sigma_n^{ii}\sum_{a\leq\lambda}w_{aa;i}^2-\frac{w_{ii}C_{\epsilon}}{\delta^2}\sum_{a>\lambda}w_{aa;i}^2,
\]
after having adapted $\epsilon$ if necessary and having chosen $\delta'$ sufficiently small
again, combining this last inequality with \eqref{app3-1} for $1\leq i\leq \lambda$ yields
\begin{align}\begin{split}
\|W\|^2(A_i+B_i+C_i+D_i-E_i)&\geq \sum_{j\neq i}\sigma_{n-1}(\kappa|j)w_{jj;i}^2-\frac{w_{ii}C_{\epsilon}}{\delta^2}\sum_{j>\lambda}w_{jj;i}^2\\
			&\geq \sum_{j>\lambda}\left(\sigma_{n-1}(\kappa|j)-\frac{w_{ii}C_{\epsilon}}{\delta^2}\right)w_{jj;i}^2,
\end{split}\end{align}
which is non-negative for small $\delta'$ for the same reason as in \eqref{app3-2}. This completes the proof.

\end{proof}

\begin{corollary}\label{Alternative}
There exist positive numbers $\delta_1,\dots,\delta_n,$ depending only on the dimension, on $\epsilon$ and on the bounds for the Gauss curvature, such that either
\begin{equation}
\kappa_i\geq \delta_i\kappa_1\quad\forall 1\leq i\leq n
\end{equation}
or
\begin{equation}
A_i+B_i+C_i+D_i-E_i\geq 0\quad\forall 1\leq i\leq n.
\end{equation}
\end{corollary}

\section*{Acknowledgement}

\bibliographystyle{amsplain}
\begin{thebibliography}{10}
\bibitem{1}J. Ai, K.S. Chou, and J. Wei, ``Self-similar solutions for the anisotropic affine curve shortening problem." Calc. Var. Partial Differential Equations 13(2001): 311--337.
\bibitem{A2} A.D. Aleksandrov, ``On the theory of mixed volumes. III. extensions of two theorems of Minkowski on convex polyhedra to arbitrary convex bodies." Mat. Sb. (N.S.) 3(1939): 167--174.
\bibitem{A3} A.D. Aleksandrov, ``Smoothness of the convex surface of bounded Gaussian curvature." C.R. (Dokl.) Acad. Sci. URSS 36(1942): 195--199.
\bibitem{A4} A.D. Aleksandrov, ``On the surface area measure of convex bodies." Mat. Sb. (N.S.) 6(1983): 27--46.
\bibitem{AGN} B. Andrews, P. Guan, and L. Ni, ``Flow by powers of the Gauss curvature." Adv. in Math. 299(2016): 174--201.
\bibitem{AnCrys} B. Andrews, ``Singularities in crystalline curvature flows." Asian J. Math. 6(2002): 101--121.
\bibitem{Andrews 1994} B. Andrews, ``Contraction of convex hypersurfaces in Euclidean space." Calc. Var. Partial Differential Equations 2(1994): 151--171.
%\bibitem{Andrews 1996} B. Andrews, ``Contraction of convex hypersurfaces by their affine normal." J. Differential Geom. 43(1996): 207--230.
\bibitem{Andrews 1997} B. Andrews, ``Monotone quantities and unique limits for evolving convex hypersurfaces." Int. Math. Res. Not. IMRN 1997(1997): 1001--1031.
\bibitem{Andrews 1998} B. Andrews, ``Evolving convex curves." Calc. Var. Partial Differential Equations 7(1998): 315--371.
\bibitem{Andrews 1999} B. Andrews, ``Gauss curvature flow: The fate of the rolling stones." Invent. Math. 138(1999): 151--161.
\bibitem{Andrews Ben 2000} B. Andrews, ``Motion of hypersurfaces by Gauss curvature." Pacific J. Math. 195(2000): 1--34.
\bibitem{Andrews 2003} B. Andrews, ``Classification of limiting shapes for isotropic curve flows." J. Amer. Math. Soc. 16(2003): 443--459.
\bibitem{andrewschen} B. Andrews, X. Chen, ``Surfaces moving by powers of Gauss curvature." Pure and Appl. Math. Quarterly 8(2012): 825--834.
\bibitem{27} K. B\"{o}r\"{o}czky, E. Lutwak, D. Yang, and G. Zhang, ``The logarithmic Minkowski problem." J. Amer. Math. Soc. 26(2013): 831--852.
\bibitem{Cal} E. Calabi, ``Improper affine hyperspheres of convex type and a generalization of a theorem by K. Jorgens." Michigan Math. J. 5(1958): 105--126.
\bibitem{CNS}  L. Caffarelli, L. Nirenberg, and J. Spruck, ``The Drichlet problem for nonlinear second order elliptic equations I. Monge-Amp\`{e}re equations." Comm. Pure Appl. Math. 37(1984): 369--402.
\bibitem{36}  W. Chen, ``$L_p$ Minkowski problem with not necessarily positive data." Adv. in Math. 201(2006): 77--89.
\bibitem{ChYau} S.Y. Cheng, and S.T. Yau, ``On the regularity of the solution of the $n$-dimensional Minkowski problem." Comm. Pure Appl. Math. 29(1976): 495--516.
\bibitem{Chou Wang 2000} K.S. Chou, and X.J. Wang, ``A logarithmic Gauss curvature flow and the Minkowski problem." Ann. Inst. H. Poincar\'{e} Anal. Non Lin\'{e}aire 17(2000): 733--751.
\bibitem{39} K.S. Chou, and X.J. Wang, ``The $L_p$-Minkowski problem and the Minkowski problem in centroaffine geometry." Adv. in Math. 205(2006): 33--83.
\bibitem{Bennett Chow and Robert Gulliver 1996} B. Chow, and R. Gulliver, ``Aleksandrov reflection and nonlinear evolution equations, I: The $n$-sphere and $n$-ball." Calc. Var. Partial Differential Equations 4(1996): 249--264.
\bibitem{Chow-Tsai 1996} B. Chow, and D.H. Tsai, ``Geometric expansion of convex plane curves." J. Differential Geom. 44(1996): 312--330.
\bibitem{Chow-Tsai 1997} B. Chow, and D.H. Tsai, ``Expansion of convex hypersurface by non--homogeneous functions of curvature." Asian J. Math. 1(1997): 769--784.
\bibitem{DZ}J. Dou, and  M. Zhu, ``The two dimensional $L_p$ Minkowski problem and nonlinear equations with negative exponents." Adv. in Math. 230(2012): 1209--1221.
\bibitem{FJ} W. Fenchel, and B. Jessen, ``Mengenfunktionen und konvexe korper." Danske Vid. Selskab. Mat.-fys. Medd. 16(1938): 1--31.
\bibitem{49} M. Gage, and Y. Li, ``Evolving planes curves by curvature in relative geometries I." Duke Math. J. 72(1993): 441--466.
\bibitem{50} M. Gage, and Y. Li,  ``Evolving planes curves by curvature in relative geometries II." Duke Math. J. 75(1994): 79--98.
\bibitem{Gerhardt:/2006} C. Gerhardt, ``Curvature problems." Series in Geometry and Topology, vol. 39, International Press, Sommerville (2006).
\bibitem{Gerhardt 2014} C. Gerhardt, ``Non-scale-invariant inverse curvature flows in Euclidean space." Calc. Var. Partial Differential Equations 49(2014): 471--489.
\bibitem{GN} P. Guan, and L. Ni, ``Entropy and a convergence theorem for Gauss curvature flow in high dimension."  to appear in J. Eur. Math. Soc. (JEMS), arXiv:1306.0625 (2013).
\bibitem{51}  P. Guan and C.S. Lin, ``On equation $\det(u_{ij} + \delta_{ij}u) = u^p f$." preprint No. 2000-7, NCTS in Tsing-Hua University, 2000.
\bibitem{Guan}P. Guan, C. Ren, and Z. Wang, ``Global $C^2$ estimates for convex solutions of curvature equations." Comm. Pure Appl. Math. Vol. LXVIII, (2015): 1287--1325.
\bibitem{QL}Y. Huang, Yong, and Q. Lu, ``On the regularity of the $L_p$ Minkowski problem." Adv. in Appl. Math. 50(2013): 268--280.
\bibitem{Ivaki 2014-gauss} M.N. Ivaki, ``Deforming a hypersurface by Gauss curvature and support function." http://arxiv.org/pdf/1501.05456v5.pdf.
\bibitem{Ivaki-Proc} M.N. Ivaki, ``An application of dual convex bodies to the inverse Gauss curvature flow." Proc. Amer. Math. Soc. 143(2015): 1257--1271.
\bibitem{Ivaki-Stancu} M.N. Ivaki, and A. Stancu, ``Volume preserving centro-affine normal flows." Comm. Anal. Geom., 21(2013): 671--685.
\bibitem{72}  M. Jiang, L. Wang, and J. Wei, ``$2\pi$-periodic self-similar solutions for the anistropic affine curve shortening problem." Calc. Var. Partial Differential Equations 41(2011): 535--565.
\bibitem{jiang} M.Y. Jiang, ``Remarks on the 2-dimensional $L_p$-Minkowski problem." Adv. Nonlinear Stud. 10(2010): 297--313.
\bibitem{Le1}H. Lewy, ``On the existence of a closed convex surface realizing a given Riemannian metric." Proc. Nat. Acad. Sci. USA 24(1938): 104--106.
\bibitem{Le2}H. Lewy, ``On differential geometry in the large, I (Minkowski's problem)." Trans. Amer. Math. Soc. 43(1938): 258--270.
\bibitem{Lu1} E. Lutwak, ``The Brunn-Minkowski-Firey theory I: mixed volumes and the Minkowski problem." J. Differential Geom. 38(1993): 131--150.
\bibitem{Lu2} E. Lutwak, ``The Brunn-Minkowski-Firey theory II: affine and geominimal surface areas." Adv. in Math. 118(1996): 244--294.
\bibitem{LuO} E. Lutwak, and V. Oliker, ``On the regularity of solutions to a generalization of the Minkowski problem." J. Differential Geom. 41(1995): 227--246.
\bibitem{LYZ} E. Lutwak, D. Yang,  and G. Zhang, ``On the $L_p$-Minkowski problem." Trans. Amer. Math. Soc. 356(2004): 4359--4370.
\bibitem{79}  J. Lu, and X.J. Wang, ``Rotationally symmetric solutions to the $L_p$-Minkowski problem." J. Differential Equations 254(2013), 983--1005.
\bibitem{James A. McCoy 2003} J. McCoy, ``The surface area preserving mean curvature flow." Asian J. Math. 7(2003): 7--30.
\bibitem{M1} H. Minkowski, ``Allgemeine Lehrs\"{a}tze \"{u}ber die konvexen Polyeder." Nachr. Ges. Wiss. G\"{o}ttingen (1897): 198--219.
\bibitem{M2} H. Minkowski, ``Volumen und Oberflache." Math. Ann. 57(1903): 447--495.
\bibitem{N}  L. Nirenberg, ``The Weyl and Minkowski problems in differential geometry in the large." Comm. Pure Appl. Math. 6(1953): 337--394.
\bibitem{P1} A.V. Pogorelov, ``Regularity of a convex surface with given Gaussian curvature." Mat. Sb. 31(1952): 88--103 (Russian).
\bibitem{P2} A.V. Pogorelov, ``A regular solution of the $n$–dimensional Minkowski problem." Dokl. Akad. Nauk. SSSR 199(1971): 785--788; English transl., Soviet Math. Dokl. 12(1971): 1192--1196.
\bibitem{Schneider} R. Schneider, ``Convex bodies: the Brunn-Minkowski theory." Vol. 151. Cambridge University Press, 2014.
%\bibitem{Spruck}J. Spruck, and L. Xiao, ``A note on starshaped compact hypersurfaces with a prescribed scalar curvature in space forms." arXiv preprint arXiv:1505.01578 (2015).
\bibitem{s1}A. Stancu, ``Uniqueness of self-similar solutions for a crystalline flow." Indiana Univ. Math. J. 45(1996): 1157--1174.
\bibitem{s2}A. Stancu, ``On the number of solutions to the discrete two-dimensional $L_0$-Minkowski problem." Adv. in Math. 180(2003): 290--323.
\bibitem{104} A. Stancu, ``The discrete planar $L_0$-Minkowski problem." Adv. in Math. 167(2002): 160--174.
\bibitem{Alina 2012} A. Stancu, ``Centro-affine invariants for smooth convex bodies."  Int. Math. Res. Not. IMRN 2012(2012): 2289--2320.
\bibitem{Tso} Tso, Kaising ``Deforming a hypersurface by its {G}auss-{K}ronecker curvature'' Comm. Pure Appl. Math 38 No. 6 (1985): 867--882.
\bibitem{Krylov-Safonov} N.V Krylov, and M.V. Safonov, ``Certain properties of parabolic equations with measurable coefficients." Izv. Akad. Nauk SSSR Ser. Mat. 40(1981): 161--175; English transl., Math. USSR Izv. 16(1981): 151--164.
\bibitem {Oliver 2006} O. Schn\"{u}rer, ``Surfaces expanding by the inverse Gauss curvature flow."  J. Reine Angew. Math. 600(2006): 117--134.
\bibitem{Tsai 2005} D.H. Tsai, ``Behavior of the gradient for solutions of parabolic equations on the circle." Calc. Var. Partial Differential Equations 23(2005): 251--270.
\bibitem{110} V. Umanskiy, ``On the solvability of the two-dimensional $L_p$-Minkowski problem." Adv. in Math. 225(2010): 3214--3228.
\bibitem{U1} J. Urbas, ``Complete noncompact self-similar solutions of Gauss curvature flows I. Positive powers." Math. Ann. 311(1998): 251--274.
\bibitem{U2} J. Urbas, ``Complete noncompact self-similar solutions of Gauss curvature flows. II. Negative powers." Adv. Differential Equations 4(1999): 323--346.
\bibitem{Zhu1}  G. Zhu, ``The centro-affine Minkowski problem for polytopes." J. Differential Geom. 101(2015): 159--174.
\bibitem{Zhu2} G. Zhu, ``The $L_p$ Minkowski problem for polytopes for $0<p<1$." J. Func. Anal. 269(2015): 1070--1094.
\bibitem{Zhu3} G. Zhu, ``The logarithmic Minkowski problem for polytopes." Adv. in Math. 262(2014): 909--931.

\end{thebibliography}
\end{document}
---------------
\begin{thmB}
Suppose $n=2$ and let $\{K_t\}$ be a solution of (\ref{e: flow0}) on $[0,t_1]$. If $c_1\leq h_{K_t}\leq c_2$ on $[0,t_1]$, then $c_5\leq \kappa_i\le c_6$ on $[0,t_1],$ where $c_5$ and $c_6$ depend on $K_0,$ $c_1,c_2,p,\varphi$ and $t_1.$
\end{thmB}
\begin{proof}
We apply the maximum principle to the test function
\begin{align*}
\Theta:=\log \zeta -\log(h-a)+\alpha\|F\|^2
\end{align*}
where $\zeta=\sup\{w_{ij}\eta^i\eta^j:\|\eta\|=1\}$ and $0<a\leq c_1/2.$
Let $x_0$ be a point that for the first time $t_0>0$ we have $$\sup\limits_{M_0} \Theta<\sup\{\sup\limits_{M_t} \Theta: 0<t\leq t_1\}=\Theta(x_0,t_0).$$ We may choose a local orthonormal frame $\{e_1,\cdots, e_n\}$ around $x_0$ such that $w_{i}^j(x_0)=\kappa_i\delta_{ij}.$ We may assume $$w_1^1=\kappa_1=\max\{\kappa_i:1\leq i\leq n\}.$$ At $(x_0,t_0)$ we have
\begin{align*}
0\leq \partial_t\Theta\leq&\psi\dot{\Phi}\sum_i \frac{\partial S}{\partial \kappa_i}\kappa_i^2+\psi(\Phi-\dot{\Phi}S)\kappa_1+\frac{\psi\ddot{\Phi}}{\kappa_1}(\nabla_1S)^2\\
&+\frac{\psi\dot{\Phi}}{\kappa_1}S^{kl,rs}\nabla_1w_{kl}\nabla_1w_{rs}+\frac{2\dot{\Phi}}{\kappa_1}\nabla_1\psi\nabla_1S+\frac{\Phi}{\kappa_1}\nabla_1\nabla_1\psi\\
&-\psi\dot{\Phi}\frac{h}{h-a}\sum_i \frac{\partial S}{\partial \kappa_i}\kappa_i^2
-(n-1)\frac{\psi\Phi}{h-a}+\frac{\Phi}{h-a}\left(F\cdot\nabla\psi\right)\\
&-2\alpha h\psi\dot{\Phi}\sum_i \frac{\partial S}{\partial \kappa_i}+2\alpha\psi(\dot{\Phi}S-\Phi)h\\
&+\psi\dot{\Phi}\left(\sum_i \frac{\partial S}{\partial \kappa_i}\frac{(\nabla_iw_1^1)^2}{\kappa_1^2}-\sum_i \frac{\partial S}{\partial \kappa_i}\frac{(\nabla_ih)^2}{(h-a)^2}\right)
\end{align*}
and
\begin{equation}\label{eq:1}
\frac{\nabla_iw_1^1}{w_1^1}-\frac{\nabla_ih}{h-a}+2\alpha(F\cdot e_i)=0.
\end{equation}
On the other hand,
$$\|\nabla\psi\|\leq C\kappa_1\Rightarrow \left\{
                                            \begin{array}{ll}
                                              \frac{\Phi}{h-a}\left(F\cdot\nabla\psi\right)\leq C\kappa_1 & \hbox{;} \\
                                              \frac{2}{\kappa_1}\dot{\Phi}\nabla_1\psi\nabla_1S\leq\varepsilon C\frac{(\nabla_1S)^2}{\kappa_1}+C\frac{\kappa_1}{\varepsilon} & \hbox{.}
                                            \end{array}
                                          \right.$$
We choose $\varepsilon>0$ small enough such that
\[\frac{\psi}{\kappa_1}\ddot{\Phi}(\nabla_1S)^2+\varepsilon C\frac{(\nabla_1S)^2}{\kappa_1}+ C\frac{\kappa_1}{\varepsilon}\leq C\frac{\kappa_1}{\varepsilon}.\]
Also, identity (\ref{eq:1}) implies that
\[\nabla_1\nabla_1\psi\leq C(1+\kappa_1^2+\alpha\kappa_1)\Rightarrow\frac{\Phi}{\kappa_1}\nabla_1\nabla_1\psi\leq C(1+\kappa_1+\alpha).\]
Therefore
\begin{align*}
0\leq \partial_t\Theta\leq &-\frac{a\psi\dot{\Phi}}{h-a}\sum_i \frac{\partial S}{\partial \kappa_i}\kappa_i^2\\
&+(n+1)\psi\Phi\kappa_1+C\frac{\kappa_1}{\varepsilon}+\frac{\psi}{\kappa_1}\dot{\Phi}S^{kl,rs}\nabla_1w_{kl}\nabla_1w_{rs}\\
&+C(1+\kappa_1+\alpha)-2\alpha h\psi\dot{\Phi}\sum_i \frac{\partial S}{\partial \kappa_i}\\
&+\psi\dot{\Phi}\left(\sum_i \frac{\partial S}{\partial \kappa_i}\frac{(\nabla_iw_1^1)^2}{\kappa_1^2}-\sum_i \frac{\partial S}{\partial \kappa_i}\frac{(\nabla_ih)^2}{(h-a)^2}\right).
\end{align*}
We partition $\{1,\cdots,n\}$ into to sets $I,J.$
$$
\left\{
    \begin{array}{ll}
      i\in I, & \hbox{if~ $\frac{\partial S}{\partial \kappa_i}\leq n^2 \frac{\partial S}{\partial \kappa_1}$;} \\
      j\in J, & \hbox{if~$\frac{\partial S}{\partial \kappa_i}> n^2 \frac{\partial S}{\partial \kappa_1}$.}
    \end{array}
  \right.
$$
Using identity (\ref{eq:1}) and applying Young's inequality we estimate
\[\sum_I \frac{\partial S}{\partial \kappa_i}\frac{(\nabla_iw_1^1)^2}{\kappa_1^2}\leq (1+\epsilon)\sum_I\frac{\partial S}{\partial \kappa_i}\frac{(\nabla_ih)^2}{(h-a)^2}+C(1+\epsilon^{-1})\alpha^2\frac{\partial S}{\partial \kappa_1}.\]
Thus choosing $\epsilon>0$ a small enough multiple of $a^2$ yields
\begin{align*}
0\leq \partial_t\Theta\leq& -\frac{a\psi\dot{\Phi}}{C'}\sum \frac{\partial S}{\partial \kappa_i}\kappa_i^2\\
&+(n+1)\psi\Phi\kappa_1+C\frac{\kappa_1}{\varepsilon}+\frac{\psi\dot{\Phi}}{\kappa_1}S^{kl,rs}\nabla_1w_{kl}\nabla_1w_{rs}\\
&+C(1+\kappa_1+\alpha)-2\alpha h\psi\dot{\Phi}\sum \frac{\partial S}{\partial \kappa_i}\\
&+\psi\dot{\Phi}\left(\sum_J \frac{\partial S}{\partial \kappa_i}\frac{(\nabla_iw_1^1)^2}{\kappa_1^2}+C(1+\epsilon^{-1})\alpha^2\frac{\partial S}{\partial \kappa_1}\right).
\end{align*}
On the other hand, in $\mathbb{R}^3$ if $J$ is non-empty, then $J=\{2\}$; so $S$ satisfies
\begin{align*}
-\frac{1}{\kappa_1}S^{ij,kl}\nabla_1w_{ij}\nabla_1w_{kl}&\geq \frac{1}{\kappa_1}\sum_{i\neq j}\frac{\frac{\partial S}{\partial \kappa_j}-\frac{\partial S}{\partial \kappa_i}}{\kappa_i-\kappa_j}(\nabla_1w_{ij})^2\\
&\geq \frac{2}{\kappa_1}\frac{\frac{\partial S}{\partial \kappa_2}-\frac{\partial S}{\partial \kappa_1}}{\kappa_1-\kappa_2}(\nabla_2w_{11})^2\geq \frac{2}{\kappa_1^2}\frac{\partial S}{\partial \kappa_2}(\nabla_2w_{11})^2.
\end{align*}
Hence, we obtain
\begin{align*}
0\leq \partial_t\Theta\leq& -\psi\dot{\Phi}\frac{a}{C'}\sum_i \frac{\partial S}{\partial \kappa_i}\kappa_i^2
+C(1+\frac{\kappa_1}{\varepsilon}+\kappa_1+\alpha)\\
&-2\alpha h\psi\dot{\Phi}\sum_i \frac{\partial S}{\partial \kappa_i}+\alpha^2\psi\dot{\Phi}C(1+\epsilon^{-1})\frac{\partial S}{\partial \kappa_1}.
\end{align*}
In addition, we have $\sum\limits_i \frac{\partial S}{\partial \kappa_i}=\frac{1}{2}\left(\sqrt{\frac{\kappa_1}{\kappa_2}}+\sqrt{\frac{\kappa_2}{\kappa_1}}\right)\geq  \frac{1}{2}\frac{\kappa_1}{S}\geq C\kappa_1.$
Consequently, we arrive at
\begin{align*}
0\leq \partial_t\Theta&\leq
C(1+\alpha)+(\frac{C}{\varepsilon}+C-2C'\alpha)\kappa_1+(-\frac{a}{C''}\kappa_1^2+C(1+\epsilon^{-1})\alpha^2)\frac{\partial S}{\partial \kappa_1}.
\end{align*}
Taking $\alpha$ large enough such that $(\frac{C}{\varepsilon}+C-2C'\alpha)\leq-1$ completes the proof.
\end{proof}

